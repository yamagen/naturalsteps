\usepackage{hyperref}
\usepackage{pxjahyper}
\hypersetup{% hyperrefオプションリスト
 setpagesize=false,
 bookmarksnumbered=true,%
 bookmarksopen=true,%
 colorlinks=true,%
 linkcolor=black,
 citecolor=black,
 urlcolor=black,
}
\usepackage{lmodern}
\usepackage{framed}
\usepackage{wrapfig}
\usepackage{scalefnt}
\usepackage{version,url,here}	% required for `\comment' (yatex added)
\usepackage[dvipdfmx]{graphicx}	% required for `\includegraphics' (yatex added)
\usepackage{nruby}
\usepackage{natbib,url}
\usepackage{makeidx}
\usepackage{listings,plistings}
\usepackage{color}
\usepackage{xcolor}
\lstloadlanguages{[LaTeX]TeX, sh}
\colorlet{lstcolTeX}{green!50!black}
\colorlet{lstcoltext}{black}
\colorlet{lstcolshell}{blue!50!black}
\usepackage{pdfpages}

\newcounter{marginparcntbw}[chapter]
\newcommand{\theMarginparcntbw}{$\dagger$\arabic{marginparcntbw}}
\newcommand{\Marginparbw}[2][−10pt]{%
  \stepcounter{marginparcntbw}%
  \textsuperscript{\theMarginparcntbw}%
  \protect\marginpar{\vskip#1\footnotesize%
    \textsuperscript{\theMarginparcntbw}
    {#2}\par}}


\setlength{\fboxsep}{.5zw}
\setlength{\fboxrule}{.6pt}

\lstdefinestyle{shell}{language=sh, rulecolor=\color{lstcolshell!25}}
\lstdefinestyle{TeX}{language=TeX, rulecolor=\color{lstcolTeX!25}}
\lstdefinestyle{text}{language=TeX, rulecolor=\color{lstcoltext!25}}

\lstset{% 
language={C++}, 
% backgroundcolor={\color[gray]{.85}},% 
basicstyle={\small},% 
identifierstyle={\small},% 
%commentstyle={\small\ttfamily \color[rgb]{0,0.5,0}},% 
%keywordstyle={\small\bfseries \color[rgb]{0,0,1}},% 
ndkeywordstyle={\small},% 
stringstyle={\small\ttfamily}, 
frame={tb}, 
breaklines=true, 
columns=[l]{fullflexible},% 
numbers=left,% 
xrightmargin=0zw,% 
xleftmargin=3zw,% 
numberstyle={\scriptsize},% 
stepnumber=1, 
numbersep=1zw,% 
morecomment=[l]{//}% 
} 

%\usepackage[format=hang,labelsep=colon,margin=10pt,sc,small]{caption}
\bibpunct[:\,]{(}{)}{,}{a}{}{,}
\definecolor{royalblue}{rgb}{0.0, 0.14, 0.4}
\newcommand{\colorrule}[1]{%
\begingroup\color{#1}\hrule\endgroup%
}%
\newcommand{\Colorrule}[1]{%
\begingroup\color{#1}\rule{1\textwidth}{2.4pt}\endgroup%
}%
\newcommand{\TColorrule}[1]{%
\begingroup\color{#1}\rule[2pt]{1\textwidth}{.6pt}\endgroup%
}%

\usepackage{ascmac}	% required for `\boxnote' (yatex added)
\usepackage{framed}
\usepackage{color}

\definecolor{lightgray}{rgb}{0.75,0.75,0.75}

\newtheorem{theo}{定理}[section]
\newtheorem{defi}{定義}[section]
\newtheorem{lemm}{補題}[section]

\makeatletter
\renewenvironment{leftbar}{%
%  \def\FrameCommand{\vrule width 3pt \hspace{10pt}}%  デフォルトの線の太さは3pt
  \def\FrameCommand{\vrule width 1pt \hspace{10pt}}% 
  \MakeFramed {\advance\hsize-\width \FrameRestore}}%
 {\endMakeFramed}
\makeatother

\newenvironment{redleftbar}{%
  \def\FrameCommand{\textcolor{red}{\vrule width 1pt} \hspace{10pt}}% 
  \MakeFramed {\advance\hsize-\width \FrameRestore}}%
 {\endMakeFramed}

\newenvironment{lightgrayleftbar}{%
  \def\FrameCommand{\textcolor{lightgray}{\vrule width .5zw} \hspace{10pt}}% 
  \MakeFramed {\advance\hsize-\width \FrameRestore}}%
{\endMakeFramed}


\AtBeginDvi{\special{papersize=\the\paperwidth,\the\paperheight}}

\newcommand{\mini}[2]{%
\setbox0=\hbox{\tt#1}\dp0=4pt%
\setbox1=\hbox{\tiny#2}\ht1=4pt\dp1=7pt%
\leavevmode\vtop{\offinterlineskip\box0\box1}}

\ifEnglish
\renewcommand{\lstlistlistingname}{List of Source codes}
\else
\renewcommand{\lstlistlistingname}{プログラム一覧}
\fi


\newcounter{excount}
\setcounter{excount}{0}
\newcounter{kdcount}
\setcounter{kdcount}{0}
\newcounter{ancount}
\setcounter{ancount}{0}
\newcounter{notecnt} % grammar note counter
\setcounter{notecnt}{0}
\newcounter{columncnt}
\setcounter{columncnt}{0}

\newenvironment{note}{%
\refstepcounter{notecnt}
\begin{itemize}
\ifEnglish
 \item[N.\thenotecnt]

\else
 \item[N.\thenotecnt] 
 \fi
}{%
\end{itemize}
%\vspace{1\baselineskip}
}

\newenvironment{toiquestion}{%
\refstepcounter{excount}
\begin{itemize}
\ifEnglish
 \item[Q.\theexcount]
\else
 \item[問\theexcount] 
 \fi
}{%
\end{itemize}
%\vspace{1\baselineskip}
\vspace{.0\baselineskip}
}

\newenvironment{toianswer}{%
\begin{itemize}
\ifEnglish
 \item[A.\theexcount]
\else
 \item[答\theexcount] 
 \fi
}{%
\end{itemize}
\vspace{.0\baselineskip}
}

\newenvironment{instruction}{%
\begin{description}
\ifEnglish
 \item[Guide]
\else

 \item[指導法] 
 \fi
}{%
\end{description}
\vspace{1\baselineskip}
}

