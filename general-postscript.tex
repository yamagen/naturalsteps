\ifEnglish
\chapter*{Postscript}
\else
\chapter*{あとがき}
\fi

\section*{意識しないものを調べる仕事}

言語はまばたきのように意識してもできるが、日常では意識しないで使うものである。
意識しないということは、その実情について知らないか、不明である可能性がある。
身近にあるものなのに知らないのである。
しかし、身近にあるけれども知らないものはこの世の中にいくらでもある。

言語学概論にあたる本講義で心がけたことがある。
それは「言語学ありきの言語分析」という愚かなことをしないように、
まずは言語の現象を見つめ、「なぜだ」と考えることである。
そして、その考える行為そのものが言語学であるということを学生に伝えることである。

東工大の1クオータ1単位科目は7.5回(1週1回90分)である。
これだけの限られた時間で言語学の基礎をどう実践すればいいかを考えた。
圧倒的に時間が足りない。
伝えられることは限られている。
すべてをやろうとするのは明らかに無謀である。
将来的に自分で学ぶ力をつけてもらう以外に方法はない。
教科書を読み上げる授業などナンセンスである。
「東工大には言語学という科目があって、そこで言語について考え、自分で調べ、友と議論する練習をした」と、
いつの日か思い出せるような授業ができればよい、そういう授業を設計したいと考え、実施することにした。

\vspace*{1\baselineskip}

\begin{flushright}
 {\large 山 元 啓 史}\hspace*{3zw}

 {\small

 東京工業大学教授\hspace*{2zw}
 
% 大岡山キャンパスにて
 
% \today
 }
\end{flushright}


\begin{comment}
\begin{enumerate}
 \item 授業は隣の人とのおしゃべりが中心、
 \item 毎回振り返りのレポート、
 \item 試験問題を学生自らが作り、
 \item 鉛筆を転がすような試験ではなく、
 \item 本質が議論できるよう口述試験を受け、
 \item テーマを自分で選び、
 \item 決められた作法で書くことを学び、
 \item 自分で何かを明らかにする、
 \item 徹底的に能動的でなければ参加できない形式にした。
\end{enumerate}
\end{comment}
